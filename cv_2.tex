%%%%%%%%%%%%%%%%%%%%%%%%%%%%%%%%%%%%%%%%%
% Medium Length Graduate Curriculum Vitae
% LaTeX Template
% Version 1.1 (9/12/12)
%
% This template has been downloaded from:
% http://www.LaTeXTemplates.com
%
% Original author:
% Rensselaer Polytechnic Institute (http://www.rpi.edu/dept/arc/training/latex/resumes/)
%
% Important note:
% This template requires the res.cls file to be in the same directory as the
% .tex file. The res.cls file provides the resume style used for structuring the
% document.
%
%%%%%%%%%%%%%%%%%%%%%%%%%%%%%%%%%%%%%%%%%

%----------------------------------------------------------------------------------------
%	PACKAGES AND OTHER DOCUMENT CONFIGURATIONS
%----------------------------------------------------------------------------------------

\documentclass[margin, 10pt]{res} % Use the res.cls style, the font size can be changed to 11pt or 12pt here

\usepackage{helvet} % Default font is the helvetica postscript font
%\usepackage{newcent} % To change the default font to the new century schoolbook postscript font uncomment this line and comment the one above

\usepackage{hyperref}
\hypersetup{
    colorlinks=true,
    linkcolor=magenta,
    filecolor=cyan,      
    urlcolor=blue,
}

\setlength{\textwidth}{5.1in} % Text width of the document

\begin{document}

%----------------------------------------------------------------------------------------
%	NAME AND ADDRESS SECTION
%----------------------------------------------------------------------------------------

\moveleft.5\hoffset\centerline{\LARGE\bf Hyerin Cho} % Your name at the top
 
\moveleft\hoffset\vbox{\hrule width\resumewidth height 1pt}\smallskip % Horizontal line after name; adjust line thickness by changing the '1pt'

\moveleft.5\hoffset\centerline{14 Osan- ro 160beon-gil}
\moveleft.5\hoffset\centerline{Osan-si, Gyeonggi-do, 18143, Rep. of Korea}
\moveleft.5\hoffset\centerline{chyerin1996@gmail.com, riniyuni123@gist.ac.kr}
\moveleft.5\hoffset\centerline{LinkedIn: www.linkedin.com/in/hyerin-cho-gist/}
\moveleft.5\hoffset\centerline{Website: https://hyerincho.github.io}

%----------------------------------------------------------------------------------------

\begin{resume}

%----------------------------------------------------------------------------------------
%	OBJECTIVE SECTION
%----------------------------------------------------------------------------------------
 
\section{EDUCATION}  

{\bf GIST(Gwangju Institute of Science and Technology)}, Korea\\
{\sl B.S. Physics Major, Math Minor}\hfill March 2015 - Present
\begin{itemize}
\small\item[] Overall GPA: 4.0/4.5\\
Major GPA: 4.4/4.5 \footnote{The courses with PS(Physics) course code in GIST transcript, including courses taken at Caltech.}
\end{itemize}

{\bf California Institute of Technology}  \hfill September 2017 - December 2017 \\
{\sl Study Abroad Program}
\begin{itemize}
\small\item[] Ph127a Statistical Mechanics,\\
Ph77a Advanced Physics Laboratory,\\
ACM116 Introduction to Probability Models,\\
Ay20 Basic Astronomy and the Galaxy,\\
Ph103 Atomic and Molecular Physics(audit), Ph125a Quantum Mechanics(audit)
\end{itemize}

{\bf University of California, Berkeley}  \hfill June 2016 - August 2016 \\
{\sl Summer Session}
\begin{itemize}
\small\item[] Chemical Structure and Reactivity\\
The Beauty and Joy of Computing
\end{itemize}


%----------------------------------------------------------------------------------------
%	PROFESSIONAL EXPERIENCE SECTION
%----------------------------------------------------------------------------------------
 
\section{RESEARCH EXPERIENCE}

{\bf OzGrav, Swinburne University of Technology} \hfill March 2019 - Present \\
{\sl Visiting Student Intern} \\
Supervisor: Prof. Matthew Bailes
\begin{itemize}
\item[] {\sl Pipeline of Full Time Resolution Recovery for Localized ASKAP FRBs.}\\
The work is continued from the previous project from CIRA. I am working on refining the reconstruction process for FRB180924 for more accurate results, and will generalize this process to a pipeline applicable for any localized ASKAP FRBs. FRB181112 will also be reconstructed for high resolution time domain FRB science. \\
{\sl Population Analysis for Radio Telescopes.}\\
Work in progress.
\end{itemize} 

{\bf Curtin Institute of Radio Astronomy (CIRA)} \hfill December 2018 - February 2019 \\
{\sl Visiting Research Associate / Summer Studentship} \\
Supervisor: Dr. Clancy James, Professor Jean-Pierre Macquart
\begin{itemize}
\item[] {\sl Recovering the Full Time Resolution of ASKAP FRB Voltage Data.}\\
As a member of The Commensal Real-time ASKAP Fast Transients Survey (\href{http://astronomy.curtin.edu.au/research/craft/}{CRAFT}) team, I worked on inverting channelization of voltage data to retrieve its full time resolution. ASKAP's high time-resolved voltage data enables quantum optical analysis of FRBs. This new diagnosis of FRBs is expected to reveal information about the source's emission properties, and thus help solve the mystery of its origin. I took voltage data from ASKAP and recovered its full time resolution via off-line processing. In particular, I used ASKAP's first localized FRB180924, but this process will be applicable to all ASKAP's localized FRBs. Most of my work was done with Python to coherently sum signals from antennas, invert the channelization (PFB), and coherently de-disperse.
\end{itemize} 

{\bf Caltech Theoretical Astrophysics} \hfill June 2018 - August 2018 \\
{\sl Summer Undergraduate Research Fellow} \\
Supervisor: Professor Sterl Phinney
\begin{itemize}
\item[] {\sl Numerical Modeling of Time-Independent Accretion Discs with Instabilities.}\\
I wrote from scratch Python code that solves the time-independent accretion disc equations numerically. These included OPAL and Ferguson opacities, equations of state, and treatment of convection. The purpose of the project was to make realistic and general models of accretion discs covering all parameter space from Cataclysmic Variables to Active Galactic Nuclei and to investigate instabilities caused by the onset of convection and hydrogen recombination.
\end{itemize} 

{\bf GIST General Intelligence and Smart Environment Laboratory}\\
{\sl Student Intern} \hfill October 2015 - August 2017 \\
Supervisor: Professor Kin Choong Yow
\begin{itemize}
\item[] {\sl Studying Deep Learning and its applications to physics problem.}\\
I worked on a project to derive physical formulas from data by modifying Google's TensorFlow Python code.
\end{itemize} 

\section{TEACHING EXPERIENCE}

{\sl Teaching Assistant} \hfill March 2018 - June 2018 \\
GIST PS3101 Electromagnetism II (3rd year course)
\begin{itemize} \itemsep -2pt % Reduce space between items
\item[] I was selected to be the Teaching Assistant as the best student of previous year's class. I graded problem sets, midterm and final exams. I also held weekly office hours to answer questions from students.
\end{itemize}

%----------------------------------------------------------------------------------------
%	Publication & Talks SECTION
%---------------------------------------------------------------------------------------- 

\section{PUBLICATIONS \\ \& TALKS}

{\sl Caltech SURF Seminar Day} \hfill{August 2018}\\
Presentation of summer research project. \\
{\sl ICRAR Summer Student Talk} \hfill{February 2019}\\
Presentation of summer research project.

%----------------------------------------------------------------------------------------
%	Awards & Fellowships SECTION
%---------------------------------------------------------------------------------------- 

\section{AWARDS \& \\ FELLOWSHIPS}

Korea National Science and Engineering Scholarship\footnote{Full tuition covered for four years.} \hfill March 2015 - Present\\
Caltech Summer Undergraduate Research Fellowship \hfill June 2018 - August 2018\\
CIRA Summer Studentship \hfill December 2018 - February 2019\\


%----------------------------------------------------------------------------------------
%	Technology SKILLS SECTION
%----------------------------------------------------------------------------------------

\section{TECHNOLOGY \\ SKILLS} 

{\sl Programming Languages:}\\%\footnote{In order of decreasing familiarity.}\\
\begin{tabular}{rl}
    Working knowledge of:& Python, MATLAB, bash\\
    Familiar with:& C\texttt{++}, Mathematica\\
     Basic knowledge of:& Fortran, C, html  \\
\end{tabular}\\
{\sl Operating Systems:} Linux, Windows \\
{\sl Others:} MESA, TensorFlow\\
 
\section{LANGUAGE \\ PROFICIENCY} 

Korean (native) \\
English (fluent\footnote{Cumulative 3 years in the U.S. during middle school and university.}) \\
Japanese, Chinese (basic knowledge) \\
 

%----------------------------------------------------------------------------------------
%	Other ACTIVITIES SECTION
%----------------------------------------------------------------------------------------

\section{OTHER \\ ACTIVITIES} 

GIST student ambassador, {\sl Member} \hfill March 2015 - December 2015\\
GIST student council, {\sl Member} \hfill June 2015 - February 2016\\
GIST student ambassador, {\sl Vice President} \hfill December 2015 - December 2016 \\
MESA\footnote{Modules for Experiments in Stellar Astrophysics} Summer School, {\sl Student} \hfill August 2018 \\
Palomar Observatory observing proposal accepted for one night \hfill August 2018
\begin{itemize}
\item[] Spectroscopic follow-up observation of several short period binaries discovered with ZTF
\end{itemize}

%----------------------------------------------------------------------------------------
%	HOBBIES
%----------------------------------------------------------------------------------------

\section{HOBBIES} 

Hiphop dance
\begin{itemize}
\item[] I was a practice director of a dance club in GIST, and I was also an instructor for a hiphop class in Caltech.
\end{itemize}


%----------------------------------------------------------------------------------------

\end{resume}
\end{document}