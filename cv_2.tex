%%%%%%%%%%%%%%%%%%%%%%%%%%%%%%%%%%%%%%%%%
% Medium Length Graduate Curriculum Vitae
% LaTeX Template
% Version 1.1 (9/12/12)
%
% This template has been downloaded from:
% http://www.LaTeXTemplates.com
%
% Original author:
% Rensselaer Polytechnic Institute (http://www.rpi.edu/dept/arc/training/latex/resumes/)
%
% Important note:
% This template requires the res.cls file to be in the same directory as the
% .tex file. The res.cls file provides the resume style used for structuring the
% document.
%
%%%%%%%%%%%%%%%%%%%%%%%%%%%%%%%%%%%%%%%%%

%----------------------------------------------------------------------------------------
%	PACKAGES AND OTHER DOCUMENT CONFIGURATIONS
%----------------------------------------------------------------------------------------

\documentclass[margin, 10pt]{res} % Use the res.cls style, the font size can be changed to 11pt or 12pt here

\usepackage{helvet} % Default font is the helvetica postscript font
%\usepackage{newcent} % To change the default font to the new century schoolbook postscript font uncomment this line and comment the one above

%% Enable block commenting
\usepackage[]{verbatim}

\usepackage{hyperref}
\hypersetup{
    colorlinks=true,
    linkcolor=magenta,
    filecolor=cyan,      
    urlcolor=blue,
}

\newcounter{daggerfootnote}
\newcommand*{\daggerfootnote}[1]{%
    \setcounter{daggerfootnote}{\value{footnote}}%
    \renewcommand*{\thefootnote}{\fnsymbol{footnote}}%
    \footnote[2]{#1}%
    \setcounter{footnote}{\value{daggerfootnote}}%
    \renewcommand*{\thefootnote}{\arabic{footnote}}%
    }
\usepackage{url}

\setlength{\textwidth}{5.1in} % Text width of the document

\begin{document}

%----------------------------------------------------------------------------------------
%	NAME AND ADDRESS SECTION
%----------------------------------------------------------------------------------------

\moveleft.5\hoffset\centerline{\LARGE\bf Hyerin Cho (조혜린)} % Your name at the top
 
\moveleft\hoffset\vbox{\hrule width\resumewidth height 1pt}\smallskip % Horizontal line after name; adjust line thickness by changing the '1pt'

\moveleft.5\hoffset\centerline{14 Osan-ro 160beon-gil}
\moveleft.5\hoffset\centerline{Osan-si, Gyeonggi-do, 18143, Rep. of Korea}
\moveleft.5\hoffset\centerline{\href{mailto:hyerin.cho@cfa.harvard.edu}{hyerin.cho@cfa.harvard.edu}, \href{mailto:chyerin1996@gmail.com}{chyerin1996@gmail.com}}
\moveleft.5\hoffset\centerline{LinkedIn: \href{https://www.linkedin.com/in/hyerin-cho-astro/}{www.linkedin.com/in/hyerin-cho-astro}}
\moveleft.5\hoffset\centerline{Website: \href{https://hyerincho.com}{hyerincho.com}}

%----------------------------------------------------------------------------------------

\begin{resume}

%----------------------------------------------------------------------------------------
%	OBJECTIVE SECTION
%----------------------------------------------------------------------------------------
 
\section{EDUCATION}  
{\bf Center for Astrophysics $\vert$ Harvard \& Smithsonian} \hfill Sep. 2020 - Present\\
Ph.D. Candidate

{\bf GIST(Gwangju Institute of Science and Technology)}, {\sl cum laude} \\
{\sl B.S. Physics Major/Math Minor}\hfill Mar. 2015 - Feb. 2020
\begin{itemize}
\small\item[] Total GPA: 4.0/4.5 (3.7/4.0 U.S. scale)\\
Major GPA: 4.4/4.5 (4.0/4.0 U.S. scale) \footnote{The courses with PS(Physics) course code in GIST transcript, including courses taken at Caltech.}
\end{itemize}

{\bf California Institute of Technology}  \hfill Sep. 2017 - Dec. 2017 \\
{\sl Study Abroad Program}
\begin{itemize}
\small\item[] Total GPA: 3.9/4.3
\begin{comment}
Ph127a Statistical Mechanics,\\
Ph77a Advanced Physics Laboratory,\\
ACM116 Introduction to Probability Models,\\
Ay20 Basic Astronomy and the Galaxy,\\
Ph103 Atomic and Molecular Physics(audit), Ph125a Quantum Mechanics(audit)
\end{comment}
\end{itemize}

{\bf University of California, Berkeley}  \hfill Jun. 2016 - Aug. 2016 \\
{\sl Summer Session}
\begin{comment}
\begin{itemize}
\small\item[] Chemical Structure and Reactivity\\
The Beauty and Joy of Computing
\end{itemize}
\end{comment}


%----------------------------------------------------------------------------------------
%	PROFESSIONAL EXPERIENCE SECTION
%----------------------------------------------------------------------------------------
 
\section{RESEARCH EXPERIENCE}

{\bf Seoul National University} \hfill Mar. 2020 - Aug. 2020 \\
{\sl Visiting Student Intern} \\
Supervisor: Prof. Ji-hoon Kim
\begin{itemize}
\item[] {\sl Impacts of galactic perturbers on fueling the MBH with a resolution appropriate accretion model}  \\
The project investigated the impacts of various galactic perturbers such as minor galactic mergers or colliding gas clumps on fueling the central massive black holes in galaxies. For this project, an isolated galaxy was constructed and simulated as a test object to conduct such experiments on. We were able to achieve subparsec resolution with the adaptive mesh refinement cosmological simulation code \href{https://enzo-project.org/}{Enzo} and we employed a resolution appropriate accretion model for the central black hole corresponding to such extremely high resolution of the simulation.
%Due to the start of my Ph.D. program, the project is handed over to another student at SNU and is in progress, but I am still involved in the project.
\end{itemize} 

{\bf OzGrav, Swinburne University of Technology} \hfill Mar. 2019 - Jun. 2019 \\
{\sl Visiting Student Intern} \\
Supervisors: Prof. Matthew Bailes, Prof. Adam Deller, Prof. Ryan Shannon
\begin{itemize}
\item[] {\sl Localized ASKAP FRBs' high time resolution and their analysis.}\\
The work is continued from the previous project from CIRA, which is improving my software that recovers full time resolution of localized ASKAP FRB voltage data. I have generalized this software for any localized sources for ASKAP and have done high time resolution analysis. This has opened up new ways to study both FRBs and the matter that their radiation encounters on its trek through the Universe. My software and analysis led to new results about the properties of matter in the outer parts of galaxies (its “halo”), as probed by an FRB. Therefore, I am a co-author of a paper on these results, published in the journal \textit{Science} in October 2019.  % \\
%{\sl Population Analysis for Radio Telescopes.}\\
%Work in progress.
\end{itemize} 

{\bf Curtin Institute of Radio Astronomy (CIRA)} \hfill Dec. 2018 - Feb. 2019 \\
{\sl Visiting Research Associate / Summer Studentship} \\
Supervisors: Prof. Jean-Pierre Macquart, Dr. Clancy James, Dr. Ian Morrison
\begin{itemize}
\item[] {\sl Recovering the full time resolution of ASKAP FRB voltage data.}\\
As a member of The Commensal Real-time ASKAP Fast Transients Survey (\href{http://astronomy.curtin.edu.au/research/craft/}{CRAFT}) collaboration, I worked on inverting channelization of voltage data (a data processing method called polyphase filterbank inversion) to retrieve its full time resolution. Having access to ASKAP's highly resolved voltage data is expected to reveal significant information including the source's emission properties and FRBs' fine temporal and spectral structure.
%I took voltage data, which preserves phase information of light, from ASKAP and recovered its full time resolution as an off-line processing.
\end{itemize} 
% quantum optical analysis

{\bf Caltech Theoretical Astrophysics} \hfill Jun. 2018 - Aug. 2018 \\
{\sl Summer Undergraduate Research Fellow} \\
Supervisor: Prof. Sterl Phinney
\begin{itemize}
\item[] {\sl Numerical modeling of time-independent accretion discs with instabilities.}\\
I wrote Python scripts from scratch that solves the time-independent accretion disc equations numerically. These included OPAL and Ferguson opacities, equations of state, and treatment of convection. The purpose of the project was to make realistic and general models of accretion discs covering a wide parameter space from Cataclysmic Variables to Active Galactic Nuclei and to investigate instabilities caused by the onset of convection and hydrogen recombination.
\end{itemize} 

{\bf GIST General Intelligence and Smart Environment Laboratory}\\
{\sl Student Intern} \hfill Oct. 2015 - Aug. 2017 \\
Supervisor: Prof. Kin Choong Yow
\begin{itemize}
\item[] {\sl Studying deep learning and its applications to physics problems.}\\
I learned object oriented programming with C\texttt{++}, and deep learning with Google's Tensorflow. Also, I worked on a project to derive physical formulae from data based on Google's TensorFlow Python scripts.
\end{itemize} 

%----------------------------------------------------------------------------------------
%	Publication & Talks SECTION
%---------------------------------------------------------------------------------------- 

\section{PUBLICATIONS}
\href{https://arxiv.org/search/advanced?advanced=&terms-0-operator=AND&terms-0-term=Hyerin+Cho&terms-0-field=author&classification-physics_archives=all&classification-include_cross_list=include&date-filter_by=all_dates&date-year=&date-from_date=&date-to_date=&date-date_type=submitted_date&abstracts=show&size=50&order=-announced_date_first}{arXiv}, \href{https://ui.adsabs.harvard.edu/search/q=orcid\%3A\%220000-0002-2858-9481\%22&sort=date\%20desc\%2C\%20bibcode\%20desc&p_=0}{ads}
\begin{enumerate}
    \item Articles published or accepted in refereed journals\\
    %J. X. Prochaska, J. P. Macquart, M. {McQuinn}, S. {Simha}, R. M. {Shannon}, C. K. {Day}, L. {Marnoch}, S. {Ryder}, A. {Deller}, K. W. {Bannister}, S. {Bhandari}, R. {Bordoloi}, J. {Bunton}, \textbf{Hyerin Cho}, C. {Flynn}, E. K. {Mahony}, C. {Phillips}, H. {Qiu}, N. {Tejos} 2019, ``{\sl The low density and magnetization of a massive galaxy halo exposed by a fast radio burst}'', Science, 365
    - J. X. Prochaska et. al. 2019 \href{https://science.sciencemag.org/content/366/6462/231/tab-pdf}{Science}, 366, ``{\sl The low density and magnetization of a massive galaxy halo exposed by a fast radio burst}''\\
    - \textbf{Hyerin Cho} et. al. 2020 \href{https://iopscience.iop.org/article/10.3847/2041-8213/ab7824/pdf}{ApJL}, 891, ``{\sl Spectropolarimetric analysis of FRB\,181112 at microsecond resolution: Implications for Fast Radio Burst emission mechanism}''\\
    - M. W. Sammons et. al. 2020 \href{https://iopscience.iop.org/article/10.3847/1538-4357/aba7bb}{ApJ}, 900, ``{\sl First constraints on compact dark matter from Fast Radio Burst microstructure}''\\
    - S. Bhandari et. al. 2020 \href{https://iopscience.iop.org/article/10.3847/2041-8213/abb462}{ApJL}, 901, ``{\sl Limits on precursor and afterglow radio emission from a fast radio burst in a star-forming galaxy}''\\
    %\item Articles submitted to refereed journals\\
    
    %\item Articles in preparation\\
\end{enumerate}
%{\sl Rich temporal and spectral structures of ASKAP FRB\,181112 observed at nanoseconds resolution}\\
%\begin{tabular}{rl}
%{\sl (in preparation, very close to submission to ApJ Letters)}
%\end{tabular}\\
%Cho H. et. al., {\sl Reconstruction of ASKAP FRBs to the order of a nanosecond time resolution via polyphase filterbank inversion}\\
%\begin{tabular}{rl}
%{\sl (in preparation)}
%\end{tabular}\\




%----------------------------------------------------------------------------------------
%	Awards & Fellowships SECTION
%---------------------------------------------------------------------------------------- 

\section{AWARDS \& \\ FELLOWSHIPS}
\textbf{Ilju Foundation Study Abroad Scholarship} \hfill Aug. 2020 - Jul. 2024
\begin{itemize}
    \item[] A very competitive scholarship granting 30,000 USD per year for four years during a Ph.D. program. The foundation selected 6 distinguished students from all majors out of 184 applicants.
\end{itemize}
\textbf{Talent Award of Korea} (대한민국인재상) \hfill Dec. 2020
\begin{itemize}
    \item[] An award bestowed by the Minister of Education of Korea. It recognizes those individuals who are likely to become Korea's future leaders and have performed exemplary talents.
\end{itemize}
GIST Outstanding Thesis Award (우수논문상) \hfill Feb. 2020\\
GIST Future Research Talent Award (미래인재상) \hfill Feb. 2020\\
Korea National Science and Engineering Scholarship \hfill Mar. 2015 - Feb. 2020
\begin{itemize}
    \item[] A scholarship to fund full tuition for 8 semesters from Korea Student Aid Foundation, Ministry of Education \href{https://www.gov.kr/portal/service/serviceInfo/B55252900005}{(국가이공계장학금)}
\end{itemize}
CIRA Summer Studentship \hfill Dec. 2018 - Feb. 2019\\
Caltech Summer Undergraduate Research Fellowship \hfill Jun. 2018 - Aug. 2018\\


\section{TALKS}
{\bf \sl GIST SNL} (``Science" Night Live) talk on my FRB research \hfill{Oct. 2019}\\
{\bf \sl ICRAR Summer Student Talk} \hfill{Feb. 2019}\\
%Presentation of summer research project.
{\bf \sl Caltech SURF Seminar Day} \hfill{Aug. 2018}\\
%Presentation of summer research project. \\


\section{TEACHING EXPERIENCE}

{\sl Teaching Assistant} \hfill Sep. 2019 - Dec. 2019 \\
GIST MM4016 Introduction to Topology (4th-year course)
%\begin{itemize} \itemsep -2pt % Reduce space between items
%\item[] 
%\end{itemize}

{\sl Teaching Assistant} \hfill Mar. 2018 - Jun. 2018 \\
GIST PS3101 Electromagnetism II (3rd-year course)
%\begin{itemize} \itemsep -2pt % Reduce space between items
%\item[] I was selected to be the Teaching Assistant as the best student in the previous year's class. I graded problem sets, midterm and final exams. I also held weekly office hours to answer questions from students.
%\end{itemize}

\section{COURSES}
The courses taken at Caltech are denoted with \textsuperscript{\textdagger}.\\
\textit{Physics}
\begin{itemize}
    \item[] General Physics I (B+) II (A)
    \item[] Classical Mechanics (B0), Electromagnetism I (B+) II (A+), Mathematical Methods of Physics (A+), Quantum Physics I (A+) II (A+), Statistical Physics\textsuperscript{\textdagger} (A+)
    \item[] Introduction to Optics (A+), Advanced Quantum Physics (A+), Solid State Physics (A+), Basic Astronomy and the Galaxy\textsuperscript{\textdagger} (B+)
    \item[] General Physics Experiment I (A) II (A), Advanced Physics Laboratory\textsuperscript{\textdagger} (A), Experimental Physics II (A+)
\end{itemize}
\textit{Mathematics}
\begin{itemize}
    \item[] Single Variable Calculus (A+), Multivariable Calculus (A+), Introduction to Linear Algebra (A), Differential Equations (A+)
    \item[] Introduction to Probility Models\textsuperscript{\textdagger} (A), Abstract Algebra (A), Complex Analysis (A)
\end{itemize}

%----------------------------------------------------------------------------------------
%	Technology SKILLS SECTION
%----------------------------------------------------------------------------------------

\section{TECHNOLOGY \\ SKILLS} 

{\sl Programming Languages:}\\%\footnote{In order of decreasing familiarity.}\\
\begin{tabular}{rl}
    Working knowledge of:& Python, MATLAB, bash\\
    Familiar with:& C\texttt{++}, C, C shell, Mathematica\\
     Basic knowledge of:& Fortran\\
\end{tabular}\\
{\sl Operating Systems:} Linux, Windows \\
{\sl Others:} MESA, TensorFlow\\
 
\section{LANGUAGE \\ PROFICIENCY} 

Korean (native) \\
English (fluent\footnote{Cumulative 3 years living in the U.S. during middle school and university. 6 months living in Australia during research internships.}) \\
Japanese, Chinese (basic knowledge) \\
 

%----------------------------------------------------------------------------------------
%	Other ACTIVITIES SECTION
%----------------------------------------------------------------------------------------

\section{OTHER \\ ACTIVITIES} 

Overseas graduate program preparation seminar \hfill Sep. 2020
\begin{itemize}
    \item[]  I hosted a student-led seminar for GIST students who are interested in applying for graduate schools overseas but have difficulties accessing relevant information. I gathered a total of 132 students as an audience and invited more than nine GIST alums as panelists who are studying/have studied abroad at Stanford, Caltech, UCSB, etc.
\end{itemize}
Student-led study group \hfill Sep. 2019 - Dec. 2019
\begin{itemize}
    \item[] I taught General Relativity and in return was taught Fluid Dynamics. My study plan and notes can be found \href{https://docs.google.com/spreadsheets/d/148MdZeH7q8QX9hl4hwpG867U84EG6tA_QnvUUvUCtcQ/edit?usp=sharing}{here}
\end{itemize}
\href{https://www.cfa.harvard.edu/events/2019/casper/index.html}{2019 CASPER Workshop \& PIRE DSP School}, {\sl Student} \hfill Aug. 2019
\begin{itemize}
\item[] Accepted to get student travel/accommodation support to Harvard
\end{itemize}
CTPU\footnote{\href{https://ctpu.ibs.re.kr/html/ctpu_en/}{Center for Theoretical Physics of the Universe, Institute for Basic Science, Korea}}  Summer School on Cosmology and Particle Physics, {\sl Student} \hfill Jul. 2019 \\
APCTP\footnote{\href{https://www.apctp.org/main/}{Asia Pacific Center for Theoretical Physics}}-NIMS-KISTI-UNIST-KASI Summer School on Numerical Relativity and Gravitational Waves, {\sl Student} \hfill Jun. 2019 \\
Palomar Observatory observing proposal accepted for one night \hfill Aug. 2018
\begin{itemize}
\item[] Spectroscopic follow-up observation of several short period binaries discovered with ZTF
\end{itemize}
MESA\footnote{Modules for Experiments in Stellar Astrophysics} Summer School, {\sl Student} \hfill Aug. 2018 \\
GIST student ambassador, {\sl Vice President} \hfill Dec. 2015 - Dec. 2016 \\
GIST student council, {\sl Member} \hfill Jun. 2015 - Feb. 2016\\
GIST student ambassador, {\sl Member} \hfill Mar. 2015 - Dec. 2015\\


%----------------------------------------------------------------------------------------
%	Test Scores
%----------------------------------------------------------------------------------------

\section{TEST SCORES} 

Physics GRE 990/990 \\
General GRE Verbal(158/170), Quantitative(169/170), Analytical Writing(4/6) \\
TOEFL 111/120\\

%----------------------------------------------------------------------------------------
%	HOBBIES
%----------------------------------------------------------------------------------------

\section{HOBBIES} 

Hiphop dance
\begin{itemize}
\item[] I was a practice director of a dance club in GIST, and I was also an instructor for a hiphop class in Caltech.
\end{itemize}
Yoga, especially aerial yoga or pilates

%----------------------------------------------------------------------------------------

\end{resume}
\end{document}